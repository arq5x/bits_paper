\subsection{Monte Carlo Simulation Comparison}
Considering that UCSC {\tt bedIntersect} does not support Monte Carlo (MC)
simulations, modifications to the UCSC source were required.  MC simulations
consist of many rounds of uniformity distributed random interval sets, and
{\tt bedIntersect} only supports reading two from disk and reporting the
intersections.  One option to add MC simulation support was to first generate
random sets using a existing tool (e.g., BEDTools {\tt suffelBed} ), then use
the unmodified UCSC {\tt bedIntersect} to operate on those random sets.  We
determined this solution would scale because of the time spent reading and
writing these sets.  Instead we replaced the function in {\tt bedIntersect} that
read interval sets from disk with a function that generates two random interval
sets.  All subsequent intersection code remained the same.  To generate the
random interval set, first a random number was generated to determine the
chromosome.  The likelihood of selecting a particular chromosome is equal the
the proportional size of the chromosome in its genome.  The next random number
represents the staring position of the interval within the chromosome, and the
end position is determined by the fixed size given as a command line option.
The number of MC rounds is also controlled by a command line parameter.

Since most of the comparisons in an MC simulation are randomly generated, and
the number file IO operations is low , we determined that walltime was the
best timing mechanism.
